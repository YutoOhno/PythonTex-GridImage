\documentclass{article}
\usepackage[dvipdfmx]{graphicx}
\usepackage{mediabb}
\usepackage{pythontex}
\begin{document}

\begin{pycode}
import matplotlib.pyplot as plt
import numpy as np

## 図の作成
def draw_func(x, y, name):
  plt.figure(figsize=(10, 7))
  plt.plot(x, y)
  plt.savefig('{}.png'.format(name))

def sin():
  x = np.arange(-3, 3, 0.01)
  y = np.sin(x)
  draw_func(x, y, 'sin')
  
def cos():
  x = np.arange(-3, 3, 0.01)
  y = np.cos(x)
  draw_func(x, y, 'cos')

def exp():
  x = np.arange(-3, 3, 0.01)
  y = np.exp(x)
  draw_func(x, y, 'exp')

def square():
  x = np.arange(-3, 3, 0.01)
  y = np.square(x)
  draw_func(x, y, 'square')

  # ファイル名、caption, labelの指定
images = [ \
  ('sin.png', 'sin curve', 'sin'), \
  ('cos.png', 'cos curve', 'cos'), \
  ('exp.png', 'exp curve', 'exp'), \
  ('square.png', 'square curve', 'square'), \
  ]

def insert_grid_images(): # 2*N grid
  # 図の生成
  sin()
  cos()
  exp()
  square()

  # Gridで表示するためにfigure環境にminipageを作る
  head = r"\begin{figure}[ht]"
  
  # 反復部分
  content = ""
  for i in images:
    tmp = r"\
    \begin{minipage}[ht]{0.48\textwidth} \
      \centering \
      \includegraphics[width= 6 cm]{FILENAME} \
      \caption{CAPTION} \
      \label{LABEL} \
    \end{minipage} \
    \hfill"
    property = {
      "FILENAME" : i[0],
      "CAPTION" : i[1],
      "LABEL" : i[2],  
    }
    # ファイル名、caption, labelを置換
    for key, value in property.items():
      tmp = tmp.replace(key, value)

    content += tmp

  foot = r"\end{figure}"

  # 生成した文字列を返す
  return head+content+foot
\end{pycode}  


sin curve(Figure \ref{sin}), % 引用も可能
cos curve(Figure \ref{cos}), 
exp curve(Figure \ref{exp}), 
square curve(Figure \ref{square})

% ここでinsert_grid_images()を実行し、texのコードを出力する
\py{insert_grid_images()}


\end{document}